\documentclass[ignorenonframetext,plain]{beamer}
\setbeamertemplate{navigation symbols}{}

\newcommand{\vocab}{\mathcal{V}}
\newcommand{\corpus}{\mathcal{C}}
\newcommand{\pml}{p_{\textsc{ml}}}
\DeclareMathOperator*{\argmax}{argmax}
\DeclareMathOperator*{\argmin}{argmin}
\newcommand{\score}{\mathit{score}}
\newcommand{\loss}{\mathit{loss}}
\renewcommand{\vec}{\mathbf}

\title{Learning to map strings to classes}
\subtitle{Comp 542 Natural Language Processing}
\author{Deniz Yuret}

\hypersetup{colorlinks,urlcolor=red}

\begin{document}

\begin{frame}
\maketitle
\end{frame}

\begin{frame}\frametitle{Applications}
\begin{itemize}
\item
  \href{http://en.wikipedia.org/wiki/Document_classification}{Document
    classification}: Text $\rightarrow$ \{ politics, finance, $\dots$ \}
\begin{itemize}
\item See \href{http://www-nlp.stanford.edu/IR-book}{Introduction to
  Information Retrieval} Chapter 13,
  \href{http://nlp.stanford.edu/fsnlp}{Foundations of Statistical NLP}
  Chapter 16 for an introduction.
\item Some datasets: \href{http://qwone.com/~jason/20Newsgroups}{20 Newsgroups},
  \href{http://www.daviddlewis.com/resources/testcollections/reuters21578}{Reuters
    21578}, 
  \href{http://www.ai.mit.edu/projects/jmlr/papers/volume5/lewis04a/lyrl2004_rcv1v2_README.htm}{Reuters
    RCV1}, and \href{http://www.cs.cmu.edu/~webkb/}{WebKB}.
\item More datasets available at \href{http://kdd.ics.uci.edu}{UCI
  KDD}, \href{http://archive.ics.uci.edu/ml/index.html}{ML},
  \href{http://csmining.org/index.php/data.html}{CSMining},
  \href{http://www.cs.cmu.edu/~TextLearning/datasets.html}{CMU}.
\end{itemize}
\item \href{http://en.wikipedia.org/wiki/Sentiment_analysis}{Sentiment
  analysis}: Text $\rightarrow$ \{ positive, negative \}
\begin{itemize}
\item
  \href{http://www.cs.cornell.edu/home/llee/opinion-mining-sentiment-analysis-survey.html}{Survey
    book} by Pang and Lee.
\item
  \href{http://www.cs.cornell.edu/people/pabo/movie-review-data}{Movie
    review data} by Pang and Lee.
\end{itemize}
\item \href{http://en.wikipedia.org/wiki/Anti-spam_techniques}{Spam
  detection}: Text $\rightarrow$ \{ spam, regular \}
\begin{itemize}
\item
  \href{http://www.aclweb.org/aclwiki/index.php?title=Spam_filtering_datasets}{Spam
  filtering datasets} from ACL Wiki.
\item \href{http://csmining.org/index.php/data.html}{Other datasets} from
  CSMining Group.
\item \href{http://archive.ics.uci.edu/ml/datasets/Spambase}{Spambase}
  from UCI machine learning repository.
\end{itemize}
\end{itemize}
\end{frame}

\begin{frame}\frametitle{String classification models}
We need to learn a function for classification from examples:\[
f: \mathcal{X}\rightarrow\mathcal{Y}
\] where $x\in\mathcal{X}$ is a string and
$\mathcal{Y}$ is a small list of classes.  This is typically done
by first learning a scoring function:\[
\score:\, \mathcal{X}\times\mathcal{Y}\rightarrow\mathbb{R}
\] and then finding $y\in\mathcal{Y}$ that maximizes the score:\[
f(x) = \argmax_{y\in\mathcal{Y}}\score(x, y)
\]
\end{frame}

\begin{frame}\frametitle{String classification models}
\[
f(x) = \argmax_{y\in\mathcal{Y}}\score(x, y)
\]
\begin{itemize}
\item Generative: $
\score(x, y) = \Pr(x, y)
$
\item Conditional: $
\score(x, y) = \Pr(y | x)
$
\item Max-Margin: $
\score(x, y) \mbox{ is not probabilistic.}
$
\item Unsupervised: $
\score(x, y) \mbox{ is learned without (x,y) examples.}
$
\end{itemize}
\end{frame}

\section{Generative models}
\frame{\sectionpage}

\begin{frame}\frametitle{Naive Bayes: Generative Process}
To generate each (x, y) pair where $x = [x_1, x_2, \dots, x_n]$
consists of words $x_i$ from a vocabulary $\vocab$ and
$y\in\mathcal{Y}$ is a class:
\begin{itemize}
\item First pick $y\in\mathcal{Y}$ with probability $q_y$ (categorical
  distribution).
\item Then pick each word $x_i\in\vocab$ with probability\footnote{The
  length $n$ is assumed given, we could model it as well if
  desired.}\footnote{The probability of each word $x_i$ depends on the
  class $y$ but is conditionally independent of other words
  $x_{j\neq i}$.} $q_{x_i|y}$ (more categorical distributions).
\end{itemize}
\end{frame}

\begin{frame}\frametitle{Naive Bayes: Likelihood}
Given the Naive Bayes generative process, what is the probability of
generating a particular (x, y)?\[ \Pr(x, y) = q_y q_{x_1|y} q_{x_2|y}
\dots = q_y \prod q_{x_i|y}
\]
\end{frame}

\begin{frame}\frametitle{Naive Bayes: Estimation}
What are the maximum likelihood estimates for the Naive Bayes
parameters $Q$?\begin{eqnarray*}
q_y^* &=& \frac{\mbox{number of documents with class
    $y$}}{\mbox{number of documents}}\\
\\
q_{x_i|y}^* &=& \frac{\mbox{number of words $=x_i$ in class
    $y$}}{\mbox{number of words in class $y$}}
\end{eqnarray*}  
\end{frame}

\begin{frame}\frametitle{Naive Bayes: Issues}
\begin{itemize}
\item Maximum likelihood overfits: some words will have zero
  probability if never before observed with a class.
  \\ \textsl{Possible solution:} Use add-one (or other) smoothing.
  \\ \textsl{Accuracy on movie reviews:} 0.8125
\item Independence assumptions (bag-of-words view) too strong.
  \\ \textsl{Possible solution:} Use ngram models for each class.
  \\ \textsl{Accuracy on movie reviews:} 0.8450
\item Joint distribution $\Pr(x,y)$ may be difficult to learn and
  unnecessary if all we need is $\argmax\Pr(y|x)$.
  \\ \textsl{Possible solution:} Conditional model, learn
  $\Pr(y|x)$ directly.
  \\ \textsl{Accuracy on movie reviews:} 0.8565
\end{itemize}
\end{frame}

\section{Conditional models}
\frame{\sectionpage}

\begin{frame}\frametitle{Why conditional models?}
\begin{center}
\includegraphics[width=.5\textwidth]{images/bishop-fig-1-27a.pdf}
\includegraphics[width=.5\textwidth]{images/bishop-fig-1-27b.pdf}
\end{center}
\scriptsize {\bf Figure 1.27}
(\href{http://research.microsoft.com/en-us/um/people/cmbishop/prml}
{Bishop, 2006}): Example of the class-conditional densities for two
classes having a single input variable $x$ (left plot) together with
the corresponding posterior probabilities (right plot). Note that the
left-hand mode of the class-conditional density $p(x|C_1)$, shown in
blue on the left plot, has no effect on the posterior
probabilities. The vertical green line in the right plot shows the
decision boundary in $x$ that gives the minimum misclassification rate.
\end{frame}

\begin{frame}\frametitle{Why conditional models?}
\begin{itemize}
\item \url{https://class.coursera.org/nlp/lecture/131}  See the text
  classification example.
\item
  \url{http://www.cs.berkeley.edu/~klein/papers/maxent-tutorial-slides.pdf}
  See examples on sensors and stoplights.  ``Even with exactly the
  same features, changing from joint to conditional estimation
  increases performance.''  (Klein and Manning 2002, WSD using
  Senseval-1 data)
\end{itemize}
\end{frame}

\begin{frame}\frametitle{Conditional log-linear models: Likelihood}
\begin{itemize}
\item Represent $(x, y)$ pairs using a feature vector function
  $\vec{g}: \mathcal{X} \times \mathcal{Y} \rightarrow
  \mathbb{R}^d:$ \[
\vec{g}(x, y) = [ g_1(x, y), g_2(x, y), \dots, g_d(x, y) ]^T
\]
\item Define score as linear combination of feature values: \[
  \score(x, y) = \vec{w}^T \vec{g}(x, y) = \sum w_i g_i(x, y)
\]
\item Maximizing score equivalent to maximizing conditional log-linear
  likelihood:
\begin{eqnarray*}
  p_w(y|x) &=& \frac{1}{z_w(x)}\exp \vec{w}^T \vec{g}(x, y) \\
  z_w(x) &=& \sum_{y'\in\mathcal{Y}} \exp \vec{w}^T \vec{g}(x, y') \\
  \argmax_{y\in\mathcal{Y}} p_w(y|x)
    &=& \argmax_{y\in\mathcal{Y}} \exp \vec{w}^T \vec{g}(x, y) \\
    &=& \argmax_{y\in\mathcal{Y}} \vec{w}^T \vec{g}(x, y)
\end{eqnarray*}
\end{itemize}
\end{frame}

\begin{frame}\frametitle{Conditional log-linear models: What features?}
A typical feature set for string classification (inspired by Naive
Bayes parameters):\begin{itemize}
\item One binary feature for each class:\[
g_j(x, y) = 
\begin{cases}
1,& \text{if } y=j \\
0,& \text{otherwise.}
\end{cases}
\]
\item One binary feature for each word-class pair:\[
g_{ij}(x, y) = 
\begin{cases}
1,& \text{if word}_i\in x, y=j \\
0,& \text{otherwise.}
\end{cases}
\]
\end{itemize}
Typically each feature should be sensitive to the class (a feature
that only looks at $x$ would not be useful in modeling $p(y|x)$), and
possibly some properties of the input string $x$.
\end{frame}

\begin{frame}\frametitle{Conditional log-linear models: MLE}
Given training data $\{(x_1, y_1), (x_2, y_2), \dots \}$, the
maximum likelihood estimation for weights $\vec{w}$: \begin{eqnarray*}
\vec{w^*} &=& \argmax_\vec{w} \prod p_\vec{w}(y_i | x_i) \\
&=& \argmax_\vec{w} \sum \log p_\vec{w}(y_i | x_i) \\
&=& \argmax_\vec{w} \sum \vec{w}^T \vec{g}(x_i, y_i) - \log z_w(x_i)
\end{eqnarray*}
\begin{itemize}
\item There is no closed form solution.
\item The function is concave (unique maximum).
\item Algorithms like LBFGS can find the maximum quickly given the
  gradient.
\end{itemize}
\end{frame}

\begin{frame}\frametitle{Conditional log-linear models: Gradient}
\begin{eqnarray*}
\ell(\vec{w}) &=& \frac{1}{N} \sum_{i=1}^N \left[ \vec{w}^T
    \vec{g}(x_i, y_i) - \log z_w(x_i) \right] \\
\frac{\partial \ell(\vec{w})}{\partial w_j} &=& \frac{1}{N}
  \sum_{i=1}^N \left[ g_j(x_i, y_i) - \frac{\sum_{y'\in\mathcal{Y}}
      g_j(x_i, y') \exp \vec{w}^T \vec{g}(x_i, y')}{z_w(x_i)}
    \right] \\
&=& \frac{1}{N} \sum_{i=1}^N \left[ g_j(x_i, y_i) -
    \sum_{y'\in\mathcal{Y}} g_j(x_i, y') p_w(y'|x_i) \right] \\
&=& \frac{1}{N} \sum_{i=1}^N \left[ g_j(x_i, y_i) -
    \mathbb{E}_{p_\vec{w}(Y|X)} g_j(x_i ,Y) \right] \\
&=& \mathbb{E}_{\tilde{p}(X,Y)} g_j(X,Y) -
  \mathbb{E}_{p_\vec{w}(X,Y)} g_j(X,Y)
\end{eqnarray*}
where $p_\vec{w}(X,Y) = \tilde{p}(X)p_\vec{w}(Y|X)$ and
$\tilde{p}$ denotes empirical distribution (i.e. data frequency).
\end{frame}

\begin{frame}\frametitle{Conditional log-linear models: Feature expectations}
\[
\mathbb{E}_{\tilde{p}(X,Y)} g_j(X,Y) =
  \mathbb{E}_{p_\vec{w}(X,Y)} g_j(X,Y)\quad \mbox{ at maximum.}
\]
\begin{itemize}
\item The first expectation is over the empirical distribution
  $\tilde{p}$.
\item The second expectation is over the model distribution $p_w$.
\item At the $\vec{w}$ that maximizes the likelihood, the model
  expectation of each feature is equal to the empirical expectation.
\end{itemize}
\end{frame}

\begin{frame}\frametitle{Conditional log-linear models: The binary case}
\begin{itemize}
\item General conditional log-linear model:\[
  p_w(y|x) = \frac{\exp \vec{w}^T \vec{g}(x, y)}
  {\sum_{y'\in\mathcal{Y}} \exp \vec{w}^T \vec{g}(x, y')}
\]
\item With binary outputs $y\in\{+1,-1\}$:\[
  p_w(Y=+1|x) = \frac{\exp \vec{w}^T \vec{g}(x, +1)}
  {\exp \vec{w}^T \vec{g}(x, +1)+\exp \vec{w}^T \vec{g}(x, -1)}
\]
\item Define $\vec{g}(\vec{x}, +1) = \vec{x}$ and $\vec{g}(\vec{x}, -1) = \vec{0}$:\[
  p_w(Y=+1|x) =
  \frac{\exp \vec{w}^T \vec{x}}
       {\exp \vec{w}^T \vec{x}+1} =
  \frac{1}{1+\exp(-\vec{w}^T \vec{x})}
\]
\item This model is known as {\bf logistic regression}.
\item When is the most probable output +1?
\end{itemize}  
\end{frame}

\begin{frame}\frametitle{Log-linear models under other names}
\begin{itemize}
\item \textbf{Maximum entropy models:} When we represent data using a
  given set of features and ask for the probability distribution with
  maximum entropy consistent with empiricial feature expectations, the
  log-linear model gives the unique solution.  (Thus they are also
  called maximum entropy models.)
\item \textbf{Logistic regression:} (actually a classification, not a
  regression technique) is a conditional log-linear model for binary
  outputs.
\item \textbf{CRF, MEMM:} Structured conditional log-linear models (to
  be covered later).
\item \textbf{Naive Bayes:} gives the maximum likelihood solution to a
  {\em generative} log-linear model (modeling joint instead of
  conditional probability as a log-linear function of features)
  i.e. $p_w(x,y) = \frac{1}{z_w}\exp \vec{w}^T \vec{g}(x, y)$
  (see
  \href{http://www.denizyuret.com/2010/11/naive-bayes-is-joint-maximum-entropy.html}{my
    blog post on this}).
%% linear svm vs logistic regression with l1/l2 prior, l1/l2 loss?
\end{itemize}
\end{frame}

\begin{frame}\frametitle{Conditional log-linear models: Overfitting}
%% LSP pp.94
Example from (LSP, pp.94): Consider a feature $g_6$ with value 1 on a
single training example $(x_9, y_9)$ and 0 for every other $(x, y)$
pair.  The derivative of the log likelihood with respect to $w_6$
is: \begin{eqnarray*} \frac{\partial \ell(\vec{w})}{\partial w_6}
  &=& \frac{1}{N} \sum_{i=1}^N \left[ g_6(x_i, y_i) -
    \sum_{y'\in\mathcal{Y}} g_6(x_i, y') p_\vec{w}(y'|x_i) \right]
  \\ &=& \frac{1}{N} \left[g_6(x_9, y_9) - g_6(x_9, y_9)
    p_\vec{w}(y_9|x_9)\right] \\ &=& \frac{1}{N} \left[ 1 -
    p_\vec{w}(y_9|x_9) \right]
\end{eqnarray*}
This derivative can only approach 0 if $p_\vec{w}(y_9|x_9)$
approaches 1.\begin{eqnarray*}
p_\vec{w}(y_9|x_9) &=& \frac{\exp \vec{w}^T \vec{g}(x_9,y_9)}
{\sum_{y'\in\mathcal{Y}} \exp \vec{w}^T \vec{g}(x_9,y')} \\
\end{eqnarray*}
This can only happen if $w_6\rightarrow\infty$ and for every feature
$g_j$ involving a competing $y'$, $w_j\rightarrow-\infty$.
\end{frame}

\begin{frame}\frametitle{MAP estimation as a solution to overfitting}
\begin{itemize}
\item Predictive distribution (what we really want):\begin{eqnarray*}
  P(D'|D,\mathcal{H}) &=& \int P(D'|Q,\mathcal{H}) P(Q|D,\mathcal{H}) dQ \\
  \text{where } P(Q|D,\mathcal{H}) &\propto& P(D|Q,\mathcal{H}) P(Q|\mathcal{H})
\end{eqnarray*}
\item Maximum likelihood estimate (MLE):\begin{eqnarray*}
  P(D'|D,\mathcal{H}) &\approx& P(D'|Q_\text{ML}, \mathcal{H}) \\
\mbox{where } Q_\text{ML} &=& \argmax_Q P(D|Q,\mathcal{H})
\end{eqnarray*}
\item Maximum a-posteriori estimate (MAP):\begin{eqnarray*}
  P(D'|D,\mathcal{H}) &\approx& P(D'|Q_\text{MAP}, \mathcal{H}) \\
\mbox{where } Q_\text{MAP} &=& \argmax_Q P(Q|D,\mathcal{H}) \\
\mbox{ and } P(Q|D,\mathcal{H}) &\propto& P(D|Q,\mathcal{H}) P(Q|\mathcal{H})
\end{eqnarray*}
\end{itemize}
\footnotesize Note: In generative models $D$ and $D'$ stand for both
the input $X$ and the output $Y$.  In conditional models they stand
for the output $Y$ only.  The input $X$ is assumed given ($X$ is on
the condition (right) side of each probability).
% expression, next to $\mathcal{H}$).
\end{frame}

\begin{frame}\frametitle{Conditional log-linear models: Gaussian MAP} %% LSP 94-97
Gaussian prior (aka L2 regularization):\begin{eqnarray*}
p(\vec{w}|\mathcal{H}_G) &=& \prod_{j=1}^d \frac{1}{\sigma \sqrt{2\pi}}
\exp -\frac{w_j^2}{2\sigma^2} \\
\vec{w}_\text{MAP} &=& \argmax_w\, \log p(D|\vec{w},\mathcal{H}_G)
+ \log p(\vec{w}|\mathcal{H}_G) \\
&=& \argmax_w\, \ell(\vec{w}) -\frac{1}{2\sigma^2}\sum w_j^2 \\
&=& \argmax_w\, \ell(\vec{w}) - \frac{C}{2}\|\vec{w}\|_2^2 \quad\text{where }C=\frac{1}{\sigma^2}
\end{eqnarray*}
\end{frame}
\begin{frame}\frametitle{Conditional log-linear models: Laplacian MAP} %% LSP 94-97
Laplacian prior (aka L1 regularization):\begin{eqnarray*}
p(\vec{w}|\mathcal{H}_L) &=& \prod_{j=1}^d \frac{1}{2b} \exp -\frac{|w_j|}{b}\\
\vec{w}_\text{MAP} &=& \argmax_w\, \log p(D|\vec{w},\mathcal{H}_L)
+ \log p(\vec{w}|\mathcal{H}_L) \\
&=& \argmax_w\, \ell(\vec{w}) -\frac{1}{b}\sum |w_j| \\
&=& \argmax_w\, \ell(\vec{w}) - C \|\vec{w}\|_1 \quad\text{where }C=\frac{1}{b}
\end{eqnarray*}
\end{frame}

\begin{frame}\frametitle{Conditional log-linear models: L1 vs. L2} %% Hastie
\includegraphics[width=.9\textwidth]{images/hastie-fig-3-11.png}

\footnotesize {\bf Figure 3.11}
(\href{http://www-stat.stanford.edu/~tibs/ElemStatLearn}{Hastie et
  al. 2009}): $\beta_1$ and $\beta_2$ are model parameters,
$\hat{\beta}$ is the ML estimate, the red ellipses are likelihood
contours, and the solid blue shapes are the L1 (left) and L2 (right)
prior contours.  The L1 prior results in a more {\bf sparse} (with
lots of zeros) parameter vector.  The L2 prior pushes many parameters
close to zero but does not quite get rid of them.
\end{frame}

\section{Non-probabilistic models}
\frame{\sectionpage}

\begin{frame}\frametitle{A non-probabilistic view of learning} %% LSP 98-100
\begin{itemize}
\item An alternative view of machine learning replaces probabilistic
  inference with the minimization of a {\bf loss
    function}\footnote{There is quite a bit of confusing terminology
    in the literature: the loss is also referred to as the cost,
    error, or regret and its negative as reward, utility, or the
    objective function.}.
\item Let $\mathcal{H}$ denote the set of all possible predictors that
  map $\mathcal{X}\rightarrow\mathcal{Y}$.  The loss for
  $h\in\mathcal{H}$ is a measurement of the badness of the predictor
  $h$ given input $x$ when $y$ is the correct output:
  \[ \loss(x, y; h) \in \mathcal{X} \times \mathcal{Y}
  \times (\mathcal{X}\rightarrow\mathcal{Y}) \rightarrow \mathbb{R}
\]
\item Given $\mathcal{H}$ and $\loss$, the learning problem becomes:\[
  \argmin_{h\in\mathcal{H}} \mathbb{E}[\loss(X, Y; h)] +
  \textit{model-complexity}(h)
\]
\item The first term (expected loss) is known as the {\bf
  risk}\footnote{The risk is typically estimated using training data
  in which case it is known as the {\bf empirical risk}:
  $\mathbb{E}[\loss(X, Y; h)] \approx \frac{1}{N} \sum_{i=1}^N
  \loss(x_i, y_i; h)$ }, and the second term is called the {\bf
  regularization} term\footnote{The regularization term serves the
  same function as the prior in MAP.}.
\end{itemize}
\end{frame}

\begin{frame}\frametitle{Probabilistic inference $\rightarrow$ log loss}
The probabilistic learning methods we have covered can be restated in
terms of loss functions.
\begin{itemize}
\item Maximizing log likelihood in generative models can be seen as
  minimizing the {\bf log loss}:\[
\loss(x, y; h) = -\log p(x, y | h)
\]
\item Maximizing log likelihood in conditional models gives another
  form of {\bf log loss}:\[
\loss(x, y; h) = -\log p(y | x, h)
\]
\end{itemize}
\end{frame}

\begin{frame}\frametitle{MAP prior $\rightarrow$ regularization term}
\begin{itemize}
\item Gaussian prior is equivalent to L2 regularization: \[
  \argmin_\vec{w}\, \frac{1}{N} \sum_{i=1}^N -\log p(y_i|x_i,\vec{w}) + \frac{C}{2}\|\vec{w}\|_2^2
\]
\item Laplacian prior is equivalent to L1 regularization: \[
  \argmin_\vec{w}\, \frac{1}{N} \sum_{i=1}^N -\log p(y_i|x_i,\vec{w}) + C \|\vec{w}\|_1
\]
\end{itemize}
\end{frame}

\begin{frame}\frametitle{Potential problems with log loss}
\begin{eqnarray*}
\loss(x, y; h) &=& -\log p(x, y | h)\quad\text{or}\\
\loss(x, y; h) &=& -\log p(y | x, h)
\end{eqnarray*}
\begin{itemize}
\item Even if the model correctly predicts $y_i$ as the most probable
  output for $x_i$, there is still pressure to diminish the loss
  function as long as other $y\in\mathcal{Y}$ have non-zero
  probability.
\item Not all alternative $y\in\mathcal{Y}$ may be equally bad, some
  wrong answers may be preferrable to others.  This is not represented
  in log loss.
\end{itemize}
\end{frame}

\begin{frame}\frametitle{Alternative loss functions}
\begin{center}
\includegraphics[height=.5\textheight]{images/bishop-fig-7-5.pdf}
\end{center}\footnotesize
{\bf Figure 7.5}
(\href{http://research.microsoft.com/en-us/um/people/cmbishop/prml}
{Bishop, 2006}): Loss functions for binary classification $y=\pm 1$.
$z=yy'$ where $y$ is the correct output and $y'$ is the model output.
$E(z)$ gives the loss (or error) for a single instance.  The red curve
is equivalent to {\bf log-loss}
($\propto\log(1+\exp(-\vec{w}^T\vec{x}))$), the black curve is {\bf
  0-1 loss}, the blue curve is {\bf hinge loss} (used by SVM), and the
green curve is the {\bf squared error} loss (used in regression).
Each loss function may imply different optimum parameters for a given
model.
\end{frame}

\begin{frame}\frametitle{Gradient descent}
\begin{itemize}
\item We typically represent predictor $h$ in terms of some parameters
  $\vec{w}$ and search for $\vec{w}$ that minimize empirical
  risk: \[ 
  \argmin_\vec{w} \sum_{i=1}^N \loss(x_i, y_i; \vec{w})
  \]
\item {\bf Gradient descent} can be used when no closed form solution
  is available (beware of local minima): \[ \vec{w} \leftarrow
  \vec{w} - \alpha \sum_{i=1}^N \nabla \loss(x_i, y_i; \vec{w})
\]
\item {\bf Stochastic gradient descent} approximates the true gradient
  of the loss by a gradient at a single example: \[ \vec{w}
  \leftarrow \vec{w} - \alpha\, \nabla \loss(x_i, y_i; \vec{w})
\]
\end{itemize}
\end{frame}

\begin{frame}\frametitle{Perceptron Algorithm (Rosenblatt 1957,
    Collins 2002)} %% LSP 101-103
Initialize $\vec{w} \leftarrow \vec{0}$\\
For pass $t=1\dots T$, example $i=1\dots N$:\\
\begin{itemize}
\item $y' \leftarrow \argmax_{y\in\mathcal{Y}}
    \vec{w}^T\vec{g}(x_i, y)$
\item $\vec{w} \leftarrow \vec{w} 
+\vec{g}(x_i,y_i)
-\vec{g}(x_i,y')$ \\
\small (No update if $y'=y_i$.)
\end{itemize}
\vspace{5mm}
{\bf Perceptron convergence theorem:} This algorithm will find
$\vec{w}$ with zero error on training set if one exists!
\end{frame}

\begin{frame}\frametitle{Perceptron Algorithm: convergence theorem}
A training set $\{(x_1,y_1),\dots,(x_N,y_N)\}$ with diameter $R$ is
separable with margin $\delta$ if there is a $\vec{w}_*$ with
$\|\vec{w}_*\|=1$ such that: \begin{eqnarray*}
  \vec{w}_*^T\vec{g}(\vec{x}_i,y_i) - \vec{w}_*^T\vec{g}(\vec{x}_i,y'_i)
  &>& \delta \quad\text{and}\\ \|\vec{g}(\vec{x}_i, y_i) -
  \vec{g}(\vec{x}_i, y'_i)\| &<& R
\end{eqnarray*}
for all $i \in \{1,\dots,N\}$ and $y'_i \in \bar{\mathcal{Y}_i}$ where
$\bar{\mathcal{Y}_i}$ is the set of {\bf incorrect} outputs and $y_i$
is the correct output for $\vec{x}_i$.  In such a training set, the
number of mistakes $K$ made by the perceptron algorithm is bounded
by: \[ K \leq \frac{R^2}{\delta^2}
\]
\end{frame}

\begin{frame}\frametitle{Perceptron Algorithm: convergence proof}
\small
\begin{itemize}
\item Let $\vec{w}_k$ be the weights before the $k$'th mistake.
  ($\vec{w}_1 = \vec{0}$).
\item Let $i$ be the example where the $k$'th mistake is made, and
  $y'_i = \argmax_{y\in\mathcal{Y}} \vec{w}_k^T \vec{g}(\vec{x}_i,y)$
  the incorrect answer given.
\item After the update: $\vec{w}_{k+1} = \vec{w}_k + \vec{g}(\vec{x}_i,y_i) -
  \vec{g}(\vec{x}_i,y'_i)$
\item Dot product the update with $\vec{w}_*$ to get a lower bound on
  $\|\vec{w}_{k+1}\|$ :\begin{eqnarray*} 
\vec{w}_*^T \vec{w}_{k+1} &=& \vec{w}_*^T \vec{w}_k + \vec{w}_*^T
  \vec{g}(\vec{x}_i,y_i) - \vec{w}_*^T \vec{g}(\vec{x}_i,y'_i) \\
 &\geq& \vec{w}_*^T \vec{w}_k + \delta \\
 &\geq& k \delta 
  \quad\text{because } \vec{w}_1=\vec{0} \\
\|\vec{w}_{k+1}\| &\geq& k \delta 
  \quad\text{because } \|\vec{w}_*\|=1
\end{eqnarray*} 
\item Square the update to get an upper bound on
  $\|\vec{w}_{k+1}\|$: \begin{eqnarray*}
\|\vec{w}_{k+1}\|^2 &=& \|\vec{w}_k\|^2 + \|\vec{g}(\vec{x}_i,y_i) -
  \vec{g}(\vec{x}_i,y'_i)\|^2 \\
  && +\, 2 \vec{w}_k^T (\vec{g}(\vec{x}_i,y_i) -
  \vec{g}(\vec{x}_i,y'_i)) \\
&\leq& \|\vec{w}_k\|^2 + \|\vec{g}(\vec{x}_i,y_i) -
  \vec{g}(\vec{x}_i,y'_i)\|^2 \\
&\leq& \|\vec{w}_k\|^2 + R^2 \leq k R^2
\end{eqnarray*}
\item Combining the two inequalities gives: $ k^2 \delta^2 \leq k R^2
  \Rightarrow k \leq R^2 / \delta^2 $
\end{itemize}
\end{frame}

\begin{frame}\frametitle{Perceptron Algorithm: loss function}
\begin{itemize}
\item {\bf Perceptron update rule:}
\[ \vec{w} \leftarrow \vec{w} 
+\vec{g}(x_i,y_i)
-\vec{g}(x_i,y') 
\]
\item {\bf Perceptron criterion (loss function):} This update rule can
  be interpreted as stochastic gradient descent on the loss function:
% (a variant of hinge loss with the hinge at origin): 
\[
\loss_P(x_i, y_i; \vec{w}) = 
-\vec{w}^T \vec{g}(x_i,y_i)
+\max_{y'} \vec{w}^T \vec{g}(x_i,y')
\]

\item Compare this with log-loss from conditional log-linear models: \[
\loss_L(x_i, y_i; \vec{w}) = 
-\vec{w}^T \vec{g}(x_i,y_i) 
+\log \sum_{y'} \exp \vec{w}^T \vec{g}(x_i,y') 
\]

\item Compare the loss in binary case with the ones in Figure 7.5.
\end{itemize}
\end{frame}

\begin{frame}\frametitle{Perceptron Algorithm: the binary case}
\begin{itemize}
\item The general perceptron decoding rule: \[
y^* = \argmax_{y\in\mathcal{Y}} \vec{w}^T \vec{g}(\vec{x}, y)
\]
\item With binary outputs $y\in\{-1,+1\}$, define $\vec{g}(\vec{x},
  +1) = \vec{x}$ and $\vec{g}(\vec{x}, -1) = \vec{0}$ this becomes: \[ 
y^* = \begin{cases}
+1,& \text{if } \vec{w}^T \vec{x} > 0\\
-1,& \text{otherwise.}
\end{cases}
\]
\item Compare this with the most probable output from logistic regression: \[
y^* = \begin{cases}
+1,& \text{if } \frac{1}{1+\exp(-\vec{w}^T \vec{x})} > \frac{1}{2} \\
-1,& \text{otherwise.}
\end{cases}
\]
\end{itemize}
\end{frame}

\begin{frame}\frametitle{Perceptron Algorithm: convergence example (binary)}
\begin{columns}
\column{.66\textwidth}
\includegraphics[height=.4\textheight]{images/bishop-fig-4-7a.pdf}
\includegraphics[height=.4\textheight]{images/bishop-fig-4-7b.pdf}\\ 
\includegraphics[height=.4\textheight]{images/bishop-fig-4-7c.pdf}
\includegraphics[height=.4\textheight]{images/bishop-fig-4-7d.pdf}
\column{.33\textwidth}\footnotesize {\bf Figure 4.7}
(\href{http://research.microsoft.com/en-us/um/people/cmbishop/prml}
       {Bishop 2006}): 
Illustration of the convergence of the perceptron learning algorithm,
showing data points from two classes (red and blue) in a
two-dimensional feature space. The parameter vector w is shown as a
black arrow together with the corresponding decision boundary (black
line), in which the arrow points towards the decision region which
classified as belonging to the red class.
\end{columns}
\end{frame}

\begin{frame}\frametitle{Perceptron Algorithm: regularization and margin}
\begin{center}
\includegraphics[width=\textwidth]{images/largeMarginPrinciple2.pdf}
\end{center}\footnotesize
{\bf Figure 14.11} (\href{http://www.cs.ubc.ca/~murphyk/MLbook}{Murphy
  2012}): The perceptron algorithm finds an arbitrary separating
boundary if there is one and stops updating afterwards.  If there is
no separating boundary it may oscillate between different solutions.
Different ways of regularizing the perceptron include {\bf voting},
{\bf averaging}, and {\bf early stopping} (see
\href{http://aclweb.org/anthology//W/W02/W02-1001.pdf}{Collins 2002}).
Another alternative is to search for the solution with the {\bf
  maximum margin} (which leads us to SVMs).
\end{frame}

\begin{frame}\frametitle{Perceptron Algorithm: geometry and margin (binary)}
\begin{itemize}
\item Consider the rule $ y^* = 1 \text{ if } \vec{w}^T
  \vec{x} > 0$ and -1 otherwise.
\item The {\bf decision boundary} is the set of all $\vec{x}$ such
  that $\vec{w}^T \vec{x} = 0$.
\item $\vec{w}$ is perpendicular to this boundary: if $\vec{x}_a$ and
  $\vec{x}_b$ are two points that lie on the boundary, the vector
  connecting them $\vec{x}_b - \vec{x}_a$ will be perpendicular to
  $\vec{w}$: $\vec{w}(\vec{x}_b - \vec{x}_a) = \vec{w}\vec{x}_b -
  \vec{w}\vec{x}_a = 0$.
\item The distance from an arbitrary training point $\vec{x}_i$ to the
  boundary is the length of the projection of $\vec{x}_i$ on
  $\vec{w}$ \[
  \frac{|\vec{w}^T\vec{x}_i|}{\|\vec{w}\|} = \frac{\vec{w}^T\vec{x}_i y_i}{\|\vec{w}\|}
\]
(We can get rid of the absolute value for training points
  remembering $y_i=1$ for $\vec{w}^T\vec{x}_i>0$ and -1 otherwise.)
\item The {\bf margin} is the distance of the closest point to the
  boundary:\[
  \text{margin } = \min_i \frac{\vec{w}^T\vec{x}_i y_i}{\|\vec{w}\|}
\]
% \item The figure on the next slide illustrates the geometry, however
%   beware that a bias term $w_0$ has been made explicit.
\end{itemize}
\end{frame}

\begin{frame}\frametitle{Perceptron Algorithm: geometry and margin (binary)}
\begin{columns}
\column{.66\textwidth}
\includegraphics[width=\textwidth]{images/bishop-fig-4-1.pdf}
\column{.33\textwidth}\footnotesize {\bf Figure 4.1}
(\href{http://research.microsoft.com/en-us/um/people/cmbishop/prml}{Bishop 2006}): 

Illustration of the geometry of a linear discriminant function in two
dimensions. The output is $+1$ if $y(\vec{x})=\vec{w}^T\vec{x}+w_0 > 0$
and $-1$ otherwise.  The decision surface, shown in red, is
perpendicular to $\vec{w}$, and its displacement from the origin is
controlled by the bias parameter $w_0$. Also, the signed orthogonal
distance of a general point $\vec{x}$ from the decision surface is
given by $y(\vec{x})/\|w\|$.
\end{columns}
\end{frame}

\begin{frame}\frametitle{Support Vector Machines: maximizing margin} %% LSP 103-104
\begin{itemize}
\item We would like to find the $\vec{w}$ that maximizes the margin,
  i.e. the distance of the closest point to the decision boundary: \[
  \vec{w}_* = \argmax_\vec{w}\left[ \min_i \frac{\vec{w}^T\vec{x}_i
      y_i}{\|\vec{w}\| \right]}
\]
\item Any multiple of $\vec{w}_*$ will have the same decision boundary
  and margin.  Let us pick the one that makes $\vec{w}^T\vec{x}_i
  y_i=1$ for the closest point.  This turns the optimization into: \[
 \vec{w}_* = \argmax_\vec{w} \frac{1}{\|\vec{w}\|}
%  = \argmin_\vec{w}\frac{1}{2}\|\vec{w}\|^2
\quad\text{such that } \forall i ,\; \vec{w}^T\vec{x}_i y_i \geq 1
\]
\item For mathematical convenience, this is equivalently stated as: \[
 \vec{w}_* = \argmin_\vec{w}\frac{1}{2}\|\vec{w}\|^2
\quad\text{such that } \forall i ,\; \vec{w}^T\vec{x}_i y_i \geq 1
\]
\end{itemize}
\end{frame}

\begin{frame}\frametitle{Support Vector Machines: multi-class version}
\begin{itemize}
\item A multi-class generalization due to (Crammer and Singer,
  2001): \[
 \vec{w}_* = \argmin_\vec{w}\frac{1}{2}\|\vec{w}\|^2
\]
\[
\text{such that } \forall i\;, \forall y'_i\in\bar{\mathcal{Y}_i},\; 
\vec{w}^T\vec{g}(\vec{x}_i, y_i) - \vec{w}^T\vec{g}(\vec{x}_i, y'_i)
\geq \delta
\]
\end{itemize}

\end{frame}

\end{document}
